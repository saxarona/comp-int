\documentclass[titlepage, letterpaper, fleqn]{article}
\usepackage[utf8]{inputenc}
\usepackage{fancyhdr} % fancy headers, of course!
\usepackage{amsmath} % what do you think?
\usepackage{amsthm} % theorems!
\usepackage{extramarks} % more cute things
\usepackage{enumitem} % i'm not sure...
\usepackage{multicol} % multicolumn...?
\usepackage{amssymb} % more symbols
\usepackage{MnSymbol} % moar symbols?
\usepackage{booktabs} % cool looking tables
\usepackage{tikz} %venn and shizzle
\usepackage{mathrsfs} %math script for calligraphic scripting, I GUESS
\usepackage{listings}
\usepackage{hyperref}
\usepackage{tcolorbox}
\usepackage{algpseudocode}
\usepackage{csquotes}

\topmargin=-0.45in
\evensidemargin=0in
\oddsidemargin=0in
\textwidth=6.5in
\textheight=9.0in
\headsep=0.25in


%
% You should change this things~
%

\newcommand{\mahteacher}{José Carlos Ortiz Bayliss}
\newcommand{\mahclass}{Computational Intelligence}
\newcommand{\mahtitle}{\textsc{Homework 01}}
\newcommand{\mahdate}{\today}

\newcommand{\spacepls}{\vspace{5mm}}
% \newcommand{\qedpls}{\blacksquare}

\renewcommand\qedsymbol{\(\blacksquare\)}
\renewcommand{\ttdefault}{pcr} %so we can get both bold and tt fonts

\newcommand{\bigO}{\mathcal{O}} %you should be inside a math environment
%
% Header markings
%

\pagestyle{fancy}
\lhead{01170065 - Xavier Sánchez}
\chead{}
\rhead{}
\lfoot{}
\rfoot{}


\renewcommand\headrulewidth{0.4pt}
\renewcommand\footrulewidth{0.4pt}

\setlength\parindent{0pt}
% \setlength\parskip{1.5pt}
\setlength\parskip{1.5ex}


%
% Create Problem Sections (stolen directly from jdavis/latex-homework-template @ github!)
%

\newcommand{\enterProblemHeader}[1]{
\nobreak\extramarks{}{Problem \arabic{#1} continued on next page\ldots}\nobreak{}
\nobreak\extramarks{Problem \arabic{#1} (continued)}{Problem \arabic{#1} continued on next page\ldots}\nobreak{}
}

\newcommand{\exitProblemHeader}[1]{
\nobreak\extramarks{Problem \arabic{#1} (continued)}{Problem \arabic{#1} continued on next page\ldots}\nobreak{}
\stepcounter{#1}
\nobreak\extramarks{Problem \arabic{#1}}{}\nobreak{}
}

\setcounter{secnumdepth}{0}
\newcounter{partCounter}
\newcounter{homeworkProblemCounter}
\setcounter{homeworkProblemCounter}{1}
\nobreak\extramarks{Exercise \arabic{homeworkProblemCounter}}{}\nobreak{}

%Solution Environment
% \newenvironment{solution}
% {\renewcommand\qedsymbol{$\square$}\begin{proof}[Solution]}
% {\end{proof}}

% Alias for the Solution section header
\newcommand{\solution}{\vspace{2ex}\textbf{\large Solution}\vspace{2ex}}

%Alias for the new step section
\newcommand{\steppy}[1]{\textbf{\large #1}}

%
% Homework Problem Environment
%
% This environment takes an optional argument. When given, it will adjust the
% problem counter. This is useful for when the problems given for your
% assignment aren't sequential. See the last 3 problems of this template for an
% example.
%
\newenvironment{homeworkProblem}[1][-1]{
\ifnum#1>0
\setcounter{homeworkProblemCounter}{#1}
\fi
\section{Problem \arabic{homeworkProblemCounter}}
\setcounter{partCounter}{1}
\enterProblemHeader{homeworkProblemCounter}
}{
\exitProblemHeader{homeworkProblemCounter}
}

%
% Coloring of code listings
%

%New colors defined below
\definecolor{codegreen}{rgb}{0,0.6,0}
\definecolor{codegray}{rgb}{0.5,0.5,0.5}
\definecolor{codepurple}{rgb}{0.58,0,0.82}
\definecolor{backcolour}{rgb}{0.98,0.98,0.98}

%Code listing style named "mystyle"
\lstdefinestyle{mystyle}{
  backgroundcolor=\color{backcolour},
  commentstyle=\color{codegreen},
  keywordstyle=\bfseries\color{blue},
  numberstyle=\tiny\color{codegray},
  stringstyle=\color{codepurple},
  basicstyle=\footnotesize\ttfamily,
  breakatwhitespace=false,         
  breaklines=true,                 
  captionpos=b,                    
  keepspaces=true,                 
  numbers=left,                    
  numbersep=5pt,                  
  showspaces=false,                
  showstringspaces=false,
  showtabs=false,                  
  tabsize=2
}

%"mystyle" code listing set
\lstset{style=mystyle}

%
% My actual info
%

\title{
\vspace{1in}
\textbf{Tecnológico de Monterrey} \\
\vspace{0.5in}
\textmd{\mahclass} \\
\vspace{0.5in}
\large{\textit{\mahteacher}} \\
\vspace{0.5in}
\textsc{\mahtitle}\\
\author{01170065  - MIT \\
Xavier Fernando Cuauhtémoc Sánchez Díaz \\
\texttt{xavier.sanchezdz@gmail.com}}
\date{\mahdate}
}

\begin{document}

\begin{titlepage}
\maketitle
\end{titlepage}

%
% Actual document starts here~
%

\begin{homeworkProblem}

{\large \textbf{Answer the following questions}}

\spacepls

\textit{Data can change over time, in particular, we might observe different input/output relationships.
In order to account for this we can adapt our learning system to the new data by, for example, training on new examples.
If the relationship between inputs and outputs for old examples has not changed, how can we prevent a neural network from forgetting about the old data?}

\solution

Adding additional neurons, I guess?

\spacepls

\textit{Suppose that a neural network is designed for the task of classifying two different classes.
What would be a suitable choice of activation function for such a network? Justify your answer.}

\solution

The hard limit activation function, of course.

\spacepls

\textit{Suppose that we have a perceptron with a weight vector $W$ and we create a new set of weights $W' = cW$ by scaling $W$ by some positive constant $c$.
Assume that the bias is zero.
If the perceptron now uses $W'$, how would this change affect its classification decisions? Justify your answer.}

\solution

Depending on the activation function, the result may be affected.

\spacepls

\textit{Suppose that we have a perceptron with a weight vector $W$ and we create a new set of weights $W' = c + W$ by adding some constant vector $c$ to $W$.
Assume that the bias is zero.
If the perceptron now uses $W'$, how would this change affect its classification decisions? Justify your answer.}

\solution

Once again, it depends on the activation function.

\end{homeworkProblem}

\pagebreak

\begin{homeworkProblem}
{\large \textbf{Perceptron's learning rule}}

\spacepls

\textit{Solve the following classification problem with the perceptron's learning rule.
Apply each input vector in order, for as many repetitions as it takes to ensure that the problem is solved.
Assume the initial weights and bias to be $W = [0,0]$ and $b = 0$.}

\spacepls

\begin{table}[h!]
\centering
\begin{tabular}{ccc}
\toprule
$x_1$ & $x_2$ & $y$ \\ \midrule
2 & 2 & 0 \\
1 & -2 & 1 \\
-2 & 2 & 0 \\
-1 & 1 & 1 \\ \bottomrule
\end{tabular}
\end{table}

\solution

aa pls
\end{homeworkProblem}
\end{document}