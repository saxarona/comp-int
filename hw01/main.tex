\documentclass[titlepage, letterpaper, fleqn]{article}
\usepackage[utf8]{inputenc}
\usepackage{fancyhdr} % fancy headers, of course!
\usepackage[fleqn]{amsmath} % what do you think?
\usepackage{amsthm} % theorems!
\usepackage{extramarks} % more cute things
\usepackage{enumitem} % i'm not sure...
\usepackage{multicol} % multicolumn...?
\usepackage{amssymb} % more symbols
\usepackage{MnSymbol} % moar symbols?
\usepackage{booktabs} % cool looking tables
\usepackage{tikz} %venn and shizzle
\usepackage{mathrsfs} %math script for calligraphic scripting, I GUESS
\usepackage{listings}
\usepackage{hyperref}
\usepackage{tcolorbox}
\usepackage{algpseudocode}
\usepackage{csquotes}
\usepackage{pifont} % http://ctan.org/pkg/pifont


\topmargin=-0.45in
\evensidemargin=0in
\oddsidemargin=0in
\textwidth=6.5in
\textheight=9.0in
\headsep=0.25in


%
% You should change this things~
%

\newcommand{\mahteacher}{José Carlos Ortiz Bayliss}
\newcommand{\mahclass}{Computational Intelligence}
\newcommand{\mahtitle}{\textsc{Homework 01}}
\newcommand{\mahdate}{\today}

\newcommand{\spacepls}{\vspace{5mm}}
% \newcommand{\qedpls}{\blacksquare}

\renewcommand\qedsymbol{\(\blacksquare\)}
%\renewcommand{\ttdefault}{pcr} %so we can get both bold and tt fonts

\newcommand{\bigO}{\mathcal{O}} %you should be inside a math environment

\newcommand{\cmark}{\text{\ding{51}}}% checkmark
\newcommand{\xmark}{\text{\ding{55}}}% x mark

%
% Header markings
%

\pagestyle{fancy}
\lhead{01170065 - Xavier Sánchez}
\chead{}
\rhead{}
\lfoot{}
\rfoot{}


\renewcommand\headrulewidth{0.4pt}
\renewcommand\footrulewidth{0.4pt}

\setlength\parindent{0pt}
% \setlength\parskip{1.5pt}
\setlength\parskip{1.5ex}


%
% Create Problem Sections (stolen directly from jdavis/latex-homework-template @ github!)
%

\newcommand{\enterProblemHeader}[1]{
\nobreak\extramarks{}{Problem \arabic{#1} continued on next page\ldots}\nobreak{}
\nobreak\extramarks{Problem \arabic{#1} (continued)}{Problem \arabic{#1} continued on next page\ldots}\nobreak{}
}

\newcommand{\exitProblemHeader}[1]{
\nobreak\extramarks{Problem \arabic{#1} (continued)}{Problem \arabic{#1} continued on next page\ldots}\nobreak{}
\stepcounter{#1}
\nobreak\extramarks{Problem \arabic{#1}}{}\nobreak{}
}

\setcounter{secnumdepth}{0}
\newcounter{partCounter}
\newcounter{homeworkProblemCounter}
\setcounter{homeworkProblemCounter}{1}
\nobreak\extramarks{Exercise \arabic{homeworkProblemCounter}}{}\nobreak{}

%Solution Environment
\newenvironment{solution}
{\renewcommand\qedsymbol{$\square$}\begin{proof}[Solution]}
{\end{proof}}

% Alias for the Solution section header
\newcommand{\sol}{\vspace{2ex}\textbf{\large Solution}\vspace{2ex}}

%Alias for the new step section
\newcommand{\steppy}[1]{\textbf{#1}}

%
% Homework Problem Environment
%
% This environment takes an optional argument. When given, it will adjust the
% problem counter. This is useful for when the problems given for your
% assignment aren't sequential. See the last 3 problems of this template for an
% example.
%
\newenvironment{homeworkProblem}[1][-1]{
\ifnum#1>0
\setcounter{homeworkProblemCounter}{#1}
\fi
\section{Problem \arabic{homeworkProblemCounter}}
\setcounter{partCounter}{1}
\enterProblemHeader{homeworkProblemCounter}
}{
\exitProblemHeader{homeworkProblemCounter}
}

%
% Coloring of code listings
%

%New colors defined below
\definecolor{codegreen}{rgb}{0,0.6,0}
\definecolor{codegray}{rgb}{0.5,0.5,0.5}
\definecolor{codepurple}{rgb}{0.58,0,0.82}
\definecolor{backcolour}{rgb}{0.98,0.98,0.98}

%Code listing style named "mystyle"
\lstdefinestyle{mystyle}{
  backgroundcolor=\color{backcolour},
  commentstyle=\color{codegreen},
  keywordstyle=\bfseries\color{blue},
  numberstyle=\tiny\color{codegray},
  stringstyle=\color{codepurple},
  basicstyle=\footnotesize\ttfamily,
  breakatwhitespace=false,         
  breaklines=true,                 
  captionpos=b,                    
  keepspaces=true,                 
  numbers=left,                    
  numbersep=5pt,                  
  showspaces=false,                
  showstringspaces=false,
  showtabs=false,                  
  tabsize=2
}

%"mystyle" code listing set
\lstset{style=mystyle}

%
% My actual info
%

\title{
\vspace{1in}
\textbf{Tecnológico de Monterrey} \\
\vspace{0.5in}
\textmd{\mahclass} \\
\vspace{0.5in}
\large{\textit{\mahteacher}} \\
\vspace{0.5in}
\textsc{\mahtitle}\\
\author{01170065  - MIT \\
Xavier Fernando Cuauhtémoc Sánchez Díaz \\
\texttt{xavier.sanchezdz@gmail.com}}
\date{\mahdate}
}

\begin{document}

\begin{titlepage}
\maketitle
\end{titlepage}

%
% Actual document starts here~
%

\begin{homeworkProblem}

{\large \textbf{Answer the following questions}}

\spacepls

\textit{Data can change over time, in particular, we might observe different input/output relationships.
In order to account for this we can adapt our learning system to the new data by, for example, training on new examples.
If the relationship between inputs and outputs for old examples has not changed, how can we prevent a neural network from forgetting about the old data?}

\sol

Assuming that the new data is way too different from those previous examples to create a generalization, a way to deal with this is to add more layers and adjusting the weights accordingly, so that a layer classifies a subset of the examples and another takes care of another subset.
Complex models need more complex networks, of course.

\spacepls

\textit{Suppose that a neural network is designed for the task of classifying two different classes.
What would be a suitable choice of activation function for such a network? Justify your answer.}

\sol

The hard limit activation function or its symmetrical version, since we only need two values.
The perceptron as a logical gate is a nice example.
It uses hard limit and works as intended.

\spacepls

\textit{Suppose that we have a perceptron with a weight vector $W$ and we create a new set of weights $W' = cW$ by scaling $W$ by some positive constant $c$.
Assume that the bias is zero.
If the perceptron now uses $W'$, how would this change affect its classification decisions? Justify your answer.}

\sol

Nothing happens, since the perceptron uses the hard limit activation function. Hard limit assigns non-negative values to 1, and negative values to 0.
Therefore, any negative value multiplied by a positive constant will remain negative, and will still be classified as 0, while all positive values will remain positive when multiplied by a positive constant.

\spacepls

\textit{Suppose that we have a perceptron with a weight vector $W$ and we create a new set of weights $W' = c + W$ by adding some constant vector $c$ to $W$.
Assume that the bias is zero.
If the perceptron now uses $W'$, how would this change affect its classification decisions? Justify your answer.}

\sol

Assume that $c_i \gg W_i$ and that $W_i$ is a negative value and $c_i$ is a positive value.
Therefore, $W'_i = c_i + W_i$ will be positive.
A negative input $p_i$ would then be mapped to 0.

Now assume $c_i \not \gg W_i$ and therefore $W'_i$ is negative.
Hence, the same negative input $p_i$ would be mapped to 1, since $W_i p_i$ will be a positive value.
This means that using no bias when summing a constant to a vector of weights will affect the results.

\end{homeworkProblem}

\pagebreak

\begin{homeworkProblem}
{\large \textbf{Perceptron's learning rule}}

\spacepls

\textit{Solve the following classification problem with the perceptron's learning rule.
Apply each input vector in order, for as many repetitions as it takes to ensure that the problem is solved.
Assume the initial weights and bias to be $W = [0,0]$ and $b = 0$.}

\spacepls

\begin{table}[h!]
\centering
\begin{tabular}{ccc}
\toprule
$x_1$ & $x_2$ & $y$ \\ \midrule
2 & 2 & 0 \\
1 & -2 & 1 \\
-2 & 2 & 0 \\
-1 & 1 & 1 \\ \bottomrule
\end{tabular}
\end{table}

\sol

The output is calculated as $W\mathbf{p} + b$.
The learning rule updates $W$ and $b$ with $W = W + e \mathbf{p}^T$ and $b = b + e$, respectively.
The error is obtained as the difference of the expected output and the actual output.

\steppy{1st epoch}

\begin{align*}
    1 \Rightarrow \quad & [0,0]\begin{bmatrix}2 \\ 2 \end{bmatrix} + 0 = 0 \mapsto 1 \quad \xmark \quad \therefore \quad e = 0 - 1 = -1 \\
    \begin{split}
      & W = [0,0] + (-1)[2, 2] = [-2, -2] \\
      & b = 0 - 1 = -1
    \end{split} \\
    2 \Rightarrow \quad & [-2,-2]\begin{bmatrix}1 \\ -2 \end{bmatrix} - 1 = -2 + 4 -1 = 1  \mapsto 1 \quad \cmark \quad \therefore \quad e = 0\\
    3 \Rightarrow \quad & [-2,-2]\begin{bmatrix}-2 \\ 2 \end{bmatrix} - 1  = 4 - 4 - 1 = -1 \mapsto 0 \quad \cmark \quad \therefore \quad e = 0\\
    4 \Rightarrow \quad & [-2,-2]\begin{bmatrix}-1 \\ 1 \end{bmatrix} - 1  = 2 - 2 - 1 = -1 \mapsto 0 \quad \xmark \quad \therefore \quad e = 1 \\
    \begin{split}
      & W = [-2,-2] + (1)[-1, 1] = [-3, -1] \\
      & b = -1 + 1 = 0
    \end{split}
    \end{align*}

\pagebreak

\steppy{2nd epoch}

    \begin{align*}
    5 \Rightarrow \quad & [-3,-1]\begin{bmatrix}2 \\ 2 \end{bmatrix} + 0  = -6 -2 + 0 = -8 \mapsto 0 \quad \cmark \quad \therefore \quad e = 0 \\
    6 \Rightarrow \quad & [-3,-1]\begin{bmatrix}1 \\ -2 \end{bmatrix} + 0  = -3 + 2 = -1 \mapsto 0 \quad \xmark \quad \therefore \quad e = 1 \\
    \begin{split}
      & W = [-3,-1] + (1)[1, -2] = [-2, -3] \\
      & b = 0 + 1 = 1
    \end{split} \\
    7 \Rightarrow \quad & [-2,-3]\begin{bmatrix}-2 \\ 2 \end{bmatrix} + 1 = 4 - 6 + 1 = -1 \mapsto 0 \quad \cmark \quad \therefore \quad e = 0 \\
    8 \Rightarrow \quad & [-2,-3]\begin{bmatrix}-1 \\ 1 \end{bmatrix} + 1 = 2 -3 + 1 = 0 \mapsto 1 \quad \cmark \quad \therefore \quad e = 0
    \end{align*}

\steppy{3rd epoch}

    \begin{align*}
    9 \Rightarrow \quad & [-2,-3]\begin{bmatrix}2 \\ 2 \end{bmatrix} + 1 = -4 -6 + 1 = -9 \mapsto 0 \quad \cmark \quad \therefore \quad e = 0 \\
    10 \Rightarrow \quad & [-2,-3]\begin{bmatrix}1 \\ -2 \end{bmatrix} + 1 = -2 + 6 + 1 = 5 \mapsto 1 \quad \cmark \quad \therefore \quad e = 0 \qed
\end{align*}

\end{homeworkProblem}
\end{document}