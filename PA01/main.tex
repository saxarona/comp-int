\documentclass[titlepage, letterpaper, fleqn]{article}
\usepackage[utf8]{inputenc}
\usepackage{fancyhdr} % fancy headers, of course!
\usepackage{amsmath} % what do you think?
\usepackage{amsthm} % theorems!
\usepackage{extramarks} % more cute things
\usepackage{enumitem} % i'm not sure...
\usepackage{multicol} % multicolumn...?
\usepackage{amssymb} % more symbols
\usepackage{MnSymbol} % moar symbols?
\usepackage{booktabs} % cool looking tables
\usepackage{tikz} %venn and shizzle
\usepackage{mathrsfs} %math script for calligraphic scripting, I GUESS
\usepackage{listings}
\usepackage{hyperref}
\usepackage{tcolorbox}
\usepackage{algpseudocode}
\usepackage{csquotes}

\topmargin=-0.45in
\evensidemargin=0in
\oddsidemargin=0in
\textwidth=6.5in
\textheight=9.0in
\headsep=0.25in


%
% You should change this things~
%

\newcommand{\mahteacher}{José Carlos Ortiz Bayliss}
\newcommand{\mahclass}{Computational Intelligence}
\newcommand{\mahtitle}{\textsc{Programming Assignment 01}}
\newcommand{\mahdate}{\today}

\newcommand{\spacepls}{\vspace{5mm}}
% \newcommand{\qedpls}{\blacksquare}

\renewcommand\qedsymbol{\(\blacksquare\)}
\renewcommand{\ttdefault}{pcr} %so we can get both bold and tt fonts

\newcommand{\bigO}{\mathcal{O}} %you should be inside a math environment
%
% Header markings
%

\pagestyle{fancy}
\lhead{01170065 - Xavier Sánchez}
\chead{}
\rhead{}
\lfoot{}
\rfoot{}


\renewcommand\headrulewidth{0.4pt}
\renewcommand\footrulewidth{0.4pt}

\setlength\parindent{0pt}
% \setlength\parskip{1.5pt}
\setlength\parskip{1.5ex}


%
% Create Problem Sections (stolen directly from jdavis/latex-homework-template @ github!)
%

\newcommand{\enterProblemHeader}[1]{
\nobreak\extramarks{}{Problem \arabic{#1} continued on next page\ldots}\nobreak{}
\nobreak\extramarks{Problem \arabic{#1} (continued)}{Problem \arabic{#1} continued on next page\ldots}\nobreak{}
}

\newcommand{\exitProblemHeader}[1]{
\nobreak\extramarks{Problem \arabic{#1} (continued)}{Problem \arabic{#1} continued on next page\ldots}\nobreak{}
\stepcounter{#1}
\nobreak\extramarks{Problem \arabic{#1}}{}\nobreak{}
}

\setcounter{secnumdepth}{0}
\newcounter{partCounter}
\newcounter{homeworkProblemCounter}
\setcounter{homeworkProblemCounter}{1}
\nobreak\extramarks{Exercise \arabic{homeworkProblemCounter}}{}\nobreak{}

%Solution Environment
% \newenvironment{solution}
% {\renewcommand\qedsymbol{$\square$}\begin{proof}[Solution]}
% {\end{proof}}

% Alias for the Solution section header
\newcommand{\solution}{\textbf{\Large Solution}}

%Alias for the new step section
\newcommand{\steppy}[1]{\textbf{\large #1}}

%
% Homework Problem Environment
%
% This environment takes an optional argument. When given, it will adjust the
% problem counter. This is useful for when the problems given for your
% assignment aren't sequential. See the last 3 problems of this template for an
% example.
%
\newenvironment{homeworkProblem}[1][-1]{
\ifnum#1>0
\setcounter{homeworkProblemCounter}{#1}
\fi
\section{Problem \arabic{homeworkProblemCounter}}
\setcounter{partCounter}{1}
\enterProblemHeader{homeworkProblemCounter}
}{
\exitProblemHeader{homeworkProblemCounter}
}

%
% Coloring of code listings
%

%New colors defined below
\definecolor{codegreen}{rgb}{0,0.6,0}
\definecolor{codegray}{rgb}{0.5,0.5,0.5}
\definecolor{codepurple}{rgb}{0.58,0,0.82}
\definecolor{backcolour}{rgb}{0.98,0.98,0.98}

%Code listing style named "mystyle"
\lstdefinestyle{mystyle}{
  backgroundcolor=\color{backcolour},
  commentstyle=\color{codegreen},
  keywordstyle=\bfseries\color{blue},
  numberstyle=\tiny\color{codegray},
  stringstyle=\color{codepurple},
  basicstyle=\footnotesize\ttfamily,
  breakatwhitespace=false,         
  breaklines=true,                 
  captionpos=b,                    
  keepspaces=true,                 
  numbers=left,                    
  numbersep=5pt,                  
  showspaces=false,                
  showstringspaces=false,
  showtabs=false,                  
  tabsize=2
}

%"mystyle" code listing set
\lstset{style=mystyle}

%
% My actual info
%

\title{
\vspace{1in}
\textbf{Tecnológico de Monterrey} \\
\vspace{0.5in}
\textmd{\mahclass} \\
\vspace{0.5in}
\large{\textit{\mahteacher}} \\
\vspace{0.5in}
\textsc{\mahtitle}\\
\author{01170065  - MIT \\
Xavier Fernando Cuauhtémoc Sánchez Díaz \\
\texttt{xavier.sanchezdz@gmail.com}}
\date{\mahdate}
}

\begin{document}

\begin{titlepage}
\maketitle
\end{titlepage}

%
% Actual document starts here~
%

\begin{homeworkProblem}

{\large \textbf{Perceptron}}

\begin{itemize}
  \item A multiple-input perceptron was implemented.
  The perceptron receives one or more inputs and the outputs are either 0 or 1.
  \item The activation function used was the \textbf{hard limit}, which maps a value $n$ to 0 if negative, or 1 otherwise.
  \item The code was implemented in Julia, a high-level, high-performance dynamic programming language for numerical computing.
  \item A \texttt{README.md} file is included with the details needed for set up and execution.
\end{itemize}

\spacepls

{\large \textbf{Solutions}}

\spacepls

The solutions listed in this section are for 100\% accuracy unless otherwise specified.

\begin{description}
  \item[Dataset 01] \hfill \\
  $W = [0.833214 -0.381716], b = [-0.085408]$
  \item[Dataset 02] (80\% accuracy) \hfill \\
  $W = [0.844879 -0.566039], b = [-0.120398]$
  \item[Dataset 03] \hfill \\
  $W = [0.0319514 -1.07417], b =[-0.0888104]$
  \item[Dataset 04] ($\approx$ 82.35\% accuracy) \hfill \\
  $W = [0.205825 -0.322592], b = [-0.00277436]$
  \item[Dataset 05] \hfill \\
  $W = [-1.57946 0.109479; 0.16207 -2.86934], b = [-0.0806363; 1.48257]$
\end{description}

\spacepls

The perceptron is not able to fully learn the patterns on Dataset 02 and 04.
After many runs (around 5 tries of 7k runs with 5k iterations for the training phase),
it was clear that the perceptron would not be able to learn the patterns in those datasets.
This is likely caused because of the nature of the models.
Since we're using the \textit{hard limit} activation function, we're limited to learn only linear patterns, that is, classify instances in a dataset which is linearly separable.
Using different activation functions in the ``neural network'' may be beneficial for this classification.

\end{homeworkProblem}
\end{document}