\documentclass[titlepage, letterpaper, fleqn]{article}
\usepackage[utf8]{inputenc}
\usepackage{fancyhdr} % fancy headers, of course!
\usepackage[fleqn]{amsmath} % what do you think?
\usepackage{amsthm} % theorems!
\usepackage{extramarks} % more cute things
\usepackage{enumitem} % i'm not sure...
\usepackage{multicol} % multicolumn...?
\usepackage{amssymb} % more symbols
\usepackage{MnSymbol} % moar symbols?
\usepackage{booktabs} % cool looking tables
\usepackage{tikz} %venn and shizzle
\usepackage{mathrsfs} %math script for calligraphic scripting, I GUESS
\usepackage{listings}
\usepackage{hyperref}
\usepackage{tcolorbox}
\usepackage{algpseudocode}
\usepackage{csquotes}
\usepackage{pifont} % http://ctan.org/pkg/pifont


\topmargin=-0.45in
\evensidemargin=0in
\oddsidemargin=0in
\textwidth=6.5in
\textheight=9.0in
\headsep=0.25in


%
% You should change this things~
%

\newcommand{\mahteacher}{José Carlos Ortiz Bayliss}
\newcommand{\mahclass}{Computational Intelligence}
\newcommand{\mahtitle}{\textsc{Homework 03}}
\newcommand{\mahdate}{\today}

\newcommand{\spacepls}{\vspace{5mm}}
% \newcommand{\qedpls}{\blacksquare}

\renewcommand\qedsymbol{\(\blacksquare\)}
%\renewcommand{\ttdefault}{pcr} %so we can get both bold and tt fonts

\newcommand{\bigO}{\mathcal{O}} %you should be inside a math environment

\newcommand{\cmark}{\text{\ding{51}}}% checkmark
\newcommand{\xmark}{\text{\ding{55}}}% x mark

%
% Header markings
%

\pagestyle{fancy}
\lhead{01170065 - Xavier Sánchez}
\chead{}
\rhead{}
\lfoot{}
\rfoot{}


\renewcommand\headrulewidth{0.4pt}
\renewcommand\footrulewidth{0.4pt}

\setlength\parindent{0pt}
% \setlength\parskip{1.5pt}
\setlength\parskip{1.5ex}


%
% Create Problem Sections (stolen directly from jdavis/latex-homework-template @ github!)
%

\newcommand{\enterProblemHeader}[1]{
\nobreak\extramarks{}{Problem \arabic{#1} continued on next page\ldots}\nobreak{}
\nobreak\extramarks{Problem \arabic{#1} (continued)}{Problem \arabic{#1} continued on next page\ldots}\nobreak{}
}

\newcommand{\exitProblemHeader}[1]{
\nobreak\extramarks{Problem \arabic{#1} (continued)}{Problem \arabic{#1} continued on next page\ldots}\nobreak{}
\stepcounter{#1}
\nobreak\extramarks{Problem \arabic{#1}}{}\nobreak{}
}

\setcounter{secnumdepth}{0}
\newcounter{partCounter}
\newcounter{homeworkProblemCounter}
\setcounter{homeworkProblemCounter}{1}
\nobreak\extramarks{Exercise \arabic{homeworkProblemCounter}}{}\nobreak{}

%Solution Environment
\newenvironment{solution}
{\renewcommand\qedsymbol{$\square$}\begin{proof}[Solution]}
{\end{proof}}

% Alias for the Solution section header
\newcommand{\sol}{\vspace{2ex}\textbf{\large Solution}\vspace{2ex}}

%Alias for the new step section
\newcommand{\steppy}[1]{\textbf{#1}}

%
% Homework Problem Environment
%
% This environment takes an optional argument. When given, it will adjust the
% problem counter. This is useful for when the problems given for your
% assignment aren't sequential. See the last 3 problems of this template for an
% example.
%
\newenvironment{homeworkProblem}[1][-1]{
\ifnum#1>0
\setcounter{homeworkProblemCounter}{#1}
\fi
\section{Problem \arabic{homeworkProblemCounter}}
\setcounter{partCounter}{1}
\enterProblemHeader{homeworkProblemCounter}
}{
\exitProblemHeader{homeworkProblemCounter}
}

%
% Coloring of code listings
%

%New colors defined below
\definecolor{codegreen}{rgb}{0,0.6,0}
\definecolor{codegray}{rgb}{0.5,0.5,0.5}
\definecolor{codepurple}{rgb}{0.58,0,0.82}
\definecolor{backcolour}{rgb}{0.98,0.98,0.98}

%Code listing style named "mystyle"
\lstdefinestyle{mystyle}{
  backgroundcolor=\color{backcolour},
  commentstyle=\color{codegreen},
  keywordstyle=\bfseries\color{blue},
  numberstyle=\tiny\color{codegray},
  stringstyle=\color{codepurple},
  basicstyle=\footnotesize\ttfamily,
  breakatwhitespace=false,         
  breaklines=true,                 
  captionpos=b,                    
  keepspaces=true,                 
  numbers=left,                    
  numbersep=5pt,                  
  showspaces=false,                
  showstringspaces=false,
  showtabs=false,                  
  tabsize=2
}

%"mystyle" code listing set
\lstset{style=mystyle}

%
% My actual info
%

\title{
\vspace{1in}
\textbf{Tecnológico de Monterrey} \\
\vspace{0.5in}
\textmd{\mahclass} \\
\vspace{0.5in}
\large{\textit{\mahteacher}} \\
\vspace{0.5in}
\textsc{\mahtitle}\\
\author{01170065  - MIT \\
Xavier Fernando Cuauhtémoc Sánchez Díaz \\
\texttt{xavier.sanchezdz@gmail.com}}
\date{\mahdate}
}

\begin{document}

\begin{titlepage}
\maketitle
\end{titlepage}

%
% Actual document starts here~
%

\section{Applications of Fuzzy Logic}

{\large Investigate and describe at least five specific applications of fuzzy logic.
Please provide relevant details of each application and a valid reference.}

\spacepls

It has been quite a long time since the term fuzzy logic was coined, back in 1965.
Since then, many applications have been found due to its usefulness to handle fuzzy sets.

The most recurrent use is that of automation and control.
The easiest way to think of it is in a fan or any cooler.
The control mechanism works as a way to decide if the fan should turn on or off at certain \textit{levels} of heat:
\begin{itemize}
  \item If temperature is \textit{warm}, then turn on.
  \item If temperature is \textit{somewhat hot}, run at low speed.
  \item If temperature is \textit{hot}, run on full power.
  \item If temperature is \textit{cool}, then turn off.
\end{itemize}

Singhala et al.~\cite{Singhala14} present an overview on temperature control using the same principle, but they use the terms \textit{too cold}, \textit{cold}, \textit{warm}, \textit{hot} and \textit{too hot}.

A similar approach was used by Ehrlich in~\cite{Ehrlich95} almost two decades before.
The Fuzzy Logic Controller (FLC) designed by Ehrlich was implemented as a tremor reducer for joystick-based wheelchairs, so that patients with tremors do not accidentally move the wheelchair in an undesired direction.

Mahbubur and Azlan described an FLC for suspensions systems for cars, as an hybrid of both active and passive damping.
The FLC considers the velocity of the sprung mass (the car itself), and the velocity of unsprung-mass (those components not supported by the suspension), and outputs an actuator force~\cite{Mahbubur12}.

Another interesting application for an FLC is that of intelligent robots.
Chen et al~\cite{Chen17} present an FLC intended for robots to be able to imitate biological behavior such as obstacle avoidance and following walls.
For avoiding obstacles, the FLC considers \textit{Close, near} and \textit{far} as the linguistic terms for the input, while the output generates \textit{slow, medium, fast} or \textit{very fast} movement in a direction---\textit{forward, turn right} or \textit{turn left}.

However, there are numerous other ways in which to apply the ideas of fuzzy sets.
Suganthi et al~\cite{Suganthi15} present a survey on different fuzzy methods applied to renewable energy systems.
Some of the more interesting concepts mentioned in the survey are \textit{fuzzifications} of optimization techniques like the Genetic Algorithm (GA) or Particle Swarm Optimization (PSO).
The fuzzy GA is used to estimate one month ahead of solar radiation.
The fuzzy PSO technique is used to generate a FLC for frequency stabilization in photovoltaic farms.

\section{Fuzzy Logic Type 2}

{\large Investigate the main features of fuzzy logic type 2 as well as a justification of its existence.}

\spacepls

Despite the fact that fuzzy logic is used to tackle fuzzy concepts, it can't take on uncertainty.
Imagine a radar reading a Gravitational Wave (GW) coming from the depths of space. The reading shows the signal may have different wavelengths, so the gravitational wave is thought to be partially a neutron star merger and partially a binary black hole merger.
A membership function can easily denote this behavior.
However, there may be uncertainty. Is my GW detector throwing an accurate reading?

For that, an additional membership function is proposed, one which handles uncertainty of the actual membership function.
Type 1 Fuzzy Logic operates on Type 1 Fuzzy Sets (T1FS), while Type 2 Fuzzy Logic operates on Type 2 Fuzzy Sets (T2FS).
T2FS membership functions map a tuple of two variables ($x$ and $u$), to a membership function $\mu(x,u)$.
It is important to note that the $u$ ranges from 0 to 1, where 0 is complete uncertainty, and 1 is completely certainty of the membership function.

This is why T2FS functions are represented as three-dimensional plots---the $X$ axis enumerates the domain of $x$, the $y$ axis enumerate the domain of $\mu(x,u)$ and the $z$ axis enumerate the domain of $u$, i.e. [0,1].

Additionally, T1FS could be represented as T2FS when $u=0.5$, that is, the level of uncertainty is not relevant since we are not sure if the membership function is accurate or not.

The T2FS deffuzzification can be calculated as the $n$-th integral of the t-norm (usually $\min$) of all T2FS membership functions, over the ratio of the sum of the product of the membership function of each T2FS by its value, over the sum of all T2FS membership functions.~\cite{Ponce-Cruz16}.

Therefore, if using fuzzy logic type 2, the wave reader could assign a proper label for the signal read: was it strong enough to become a GW--event, or was it too weak so that it is just assigned an LVT--event label?

\bibliographystyle{abbrv}
\bibliography{biblio}

\end{document}